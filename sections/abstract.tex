
\begin{abstract}
When developing a multimedia classification system in preparation for an evaluation such as TRECVID MED or MediaEval, one typically must estimate the error on the evaluation data set from the performance on the development data set, a task made increasingly important with user-produced videos that often lack structure and have high acoustic variability. 
Statistical bootsrapping reveals the uncertainty of the estimate of a classifier's performance, yet with today's large multimedia datasets, bootstrapping applications, particularly when written in common productivity languages such as Python or Matlab, can become essentially intractable, as we later show.

To combat this, we propose here a methodology to efficiently estimate classifier performance that relies on the ability to generate distributed bootstrapping applications from serial Python. 
Utilizing an already made Domain-Specific Embedded Language (DSEL) and corresponding DSEL compiler \cite{pbirsinger2013} made with the SEJITS (Selective Embedded Just-In-Time Specialization) approach \cite{Kamil:EECS-2013-1}, we describe a bootstrapping application in Python to evaluate a multimedia classifier's performance and   automatically generate distributed code scalable to demanding data sizes and computational workloads. 
With this methodology we obtain reasonable bounds for classifier performance predictions on user-produced videos from the 2013 TRECVID MED corpus, showing that reasonbly estimating performance on such data is possible. 
We then demonstrate the newfound accessibility to such computations for non-performance programmers by exposing the orders of magnitude speedup obtained over the native Python code.
\end{abstract} 

% Statistical bootsrapping reveals the uncertainty of the estimate of a classifier's performance, yet often requires considerable computational resources, a problem compounded by today's sizable multimedia datasets.
% We show that large bootstrapping multimedia applications, particularly when written in common productivity languages such as Python or Matlab, can become essentially intractable. 